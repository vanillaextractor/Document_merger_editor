\documentclass[12pt,a4paper]{article}
\usepackage[utf8]{inputenc}
\usepackage{geometry}
\usepackage{listings}
\usepackage{xcolor}
\usepackage{hyperref}
\usepackage{graphicx}
\usepackage{fancyhdr}


% Page Layout
\geometry{
    a4paper,
    top=25mm,
    bottom=25mm,
    left=25mm,
    right=25mm,
    headheight=15pt
}

% Code highlighting configuration
\definecolor{codegreen}{rgb}{0,0.6,0}
\definecolor{codegray}{rgb}{0.5,0.5,0.5}
\definecolor{codepurple}{rgb}{0.58,0,0.82}
\definecolor{backcolour}{rgb}{0.95,0.95,0.92}

\lstdefinestyle{mystyle}{
    backgroundcolor=\color{backcolour},   
    commentstyle=\color{codegreen},
    keywordstyle=\color{magenta},
    numberstyle=\tiny\color{codegray},
    stringstyle=\color{codepurple},
    basicstyle=\ttfamily\footnotesize,
    breakatwhitespace=false,         
    breaklines=true,                 
    captionpos=b,                    
    keepspaces=true,                 
    numbers=left,                    
    numbersep=5pt,                  
    showspaces=false,                
    showstringspaces=false,
    showtabs=false,                  
    tabsize=2
}

\lstset{style=mystyle}

% Header and Footer
\pagestyle{fancy}
\fancyhf{}
\rhead{Document Merger Script Documentation}
\lhead{Version 1.0}
\cfoot{Institute of Data Engineering Analytics and Science \quad \thepage}

\title{\textbf{Document Merger Script Documentation}\\ \large Technical Manual \& Customization Guide}
\author{Institute of Data Engineering Analytics and Science}
\date{\today}

\begin{document}

\maketitle
\tableofcontents
\newpage

\section{Introduction}

The \texttt{merger.py} script is a robust automation tool designed to merge multiple documents into a single, cohesive PDF file. It supports various input formats including PDF, DOCX, TEX, and MD. 

Key features include:
\begin{itemize}
    \item \textbf{Multi-format Support}: Automatically converts DOCX, Markdown, and LaTeX to PDF before merging.
    \item \textbf{Intelligent Indexing}: Scans content for keywords defined in \texttt{keywords.txt} and generates a comprehensive index.
    \item \textbf{Smart Bookmarks}: preserving and restructuring Table of Contents (TOC) with correct page numbers.
    \item \textbf{Uniform Styling}: Applies consistent margins and headers across the final document.
\end{itemize}

\section{Prerequisites \& Installation}

Before running the script, ensure the following system dependencies are installed:

\subsection{System Requirements}
\begin{enumerate}
    \item \textbf{Python 3.8+}: The core programming language.
    \item \textbf{XeLaTeX}: Required for compiling the PDF. Usually part of a TeX distribution like TeX Live or MiKTeX.
    \item \textbf{Pandoc}: Required for converting DOCX and Markdown files.
\end{enumerate}

\subsection{Python Dependencies}
Install the required Python packages using the provided \texttt{requirements.txt}:

\begin{lstlisting}[language=bash]
pip install -r requirements.txt
\end{lstlisting}

\textbf{Core Libraries:}
\begin{itemize}
    \item \texttt{pypandoc}: Wrapper for Pandoc conversions.
    \item \texttt{PyPDF2}: For PDF manipulation and reading.
    \item \texttt{python-docx}: For interacting with Word documents.
    \item \texttt{click}: For command-line interface arguments.
\end{itemize}

\section{Usage Guide}

\subsection{Directory Structure}
The script expects the following directory structure:
\begin{verbatim}
project_root/
|-- input/              # Place all source files here (pdf, docx, tex)
|-- output/             # Final fused PDF will appear here
|-- keywords.txt        # (Optional) List of keywords for indexing
`-- scripts/
    `-- merger.py       # The main script
\end{verbatim}

\subsection{Running the Script}
Navigate to the project directory and run:

\begin{lstlisting}[language=bash]
python scripts/merger.py
\end{lstlisting}

\subsubsection{Command Line Arguments}
You can specify the output filename and document title directly:

\begin{lstlisting}[language=bash]
python scripts/merger.py --name "My_Final_Report" --title "Annual Report 2024"
\end{lstlisting}

\begin{itemize}
    \item \texttt{--name}: The filename for the output PDF (e.g., \texttt{My\_Report.pdf}).
    \item \texttt{--title}: The title text displayed on the cover page and metadata.
\end{itemize}

\section{GitHub Actions Automation}

The project includes a continuous integration workflow defined in \texttt{.github/workflows/merge.yml}. This automation ensures that the document is rebuilt whenever changes are pushed to the repository.

\subsection{Workflow Overview}
\begin{itemize}
    \item \textbf{Trigger}: Pushes to \texttt{main} or \texttt{master} branches.
    \item \textbf{Runner}: \texttt{ubuntu-latest}.
    \item \textbf{Process}:
        \begin{enumerate}
            \item Check out the repository.
            \item Install system dependencies (Pandoc, XeLaTeX, TeX Live fonts).
            \item Install Python 3.11 and dependencies from \texttt{requirements.txt}.
            \item Execute \texttt{merger.py} with automated title "NPCYF Documentation".
            \item Upload the generated PDF as a build artifact named \texttt{merged-document}.
        \end{enumerate}
\end{itemize}

\subsection{Repository}
The source code and workflow configuration are hosted at:\\
\url{https://github.com/vanillaextractor/Document_merger_editor}

\section{Code Analysis}

The script is built around the \texttt{DocumentMerger} class.

\subsection{Key Methods}

\begin{itemize}
    \item \textbf{\texttt{scan\_documents()}}: Scans the \texttt{input/} directory and registers supported files.
    
    \item \textbf{\texttt{convert\_to\_pdf(chapter)}}: Converts non-PDF files to PDF. It uses \texttt{pypandoc} for DOCX files, enforcing specific geometry settings.
    
    \item \textbf{\texttt{get\_pdf\_bookmarks(pdf\_path)}}: complex logic to parse existing bookmarks and infer chapter titles to maintain a clean TOC structure.
    
    \item \textbf{\texttt{extract\_keywords()}}: Reads \texttt{keywords.txt} and scans the text of every page to build a keyword map.
    
    \item \textbf{\texttt{create\_master\_latex()}}: Generates the final \texttt{master.tex} file which uses \texttt{pdfpages} to include all converted PDFs.
\end{itemize}

\section{Customization Guide}

This section details how to modify the script to change margins, layout, and fonts. All changes are made directly in \texttt{scripts/merger.py}.

\subsection{Changing Margins}
The margins are defined in two places. You must update both to ensure consistency.

\subsubsection{1. For DOCX/Text Conversions}
In the \texttt{convert\_to\_pdf} method (around line 75 and 88), find the \texttt{extra\_args} list. The margin is set via the \texttt{-V geometry:margin=...} option.

\begin{lstlisting}[language=python, caption=Changing Margins for Conversions]
# OLD CODE (1 inch margins)
extra_args=['--pdf-engine=xelatex', '-V', 'geometry:margin=1in']

# NEW CODE (e.g., 2cm margins)
extra_args=['--pdf-engine=xelatex', '-V', 'geometry:margin=2cm']
\end{lstlisting}

\subsubsection{2. For the Master Document}
In the \texttt{create\_master\_latex} method (around line 319), find the \texttt{\textbackslash usepackage} definition for geometry.

\begin{lstlisting}[language=python, caption=Changing Global Margins]
# OLD CODE
latex_content = r"""
\documentclass[12pt,a4paper]{book}
\usepackage{fontspec}
\usepackage[margin=1in]{geometry}  % <--- CHANGE THIS LINE
...
"""

# NEW CODE (Example: 1.5 inch left, 1 inch others)
\usepackage[left=1.5in, right=1in, top=1in, bottom=1in]{geometry}
\end{lstlisting}

\subsection{Changing Header and Footer}
The header and footer for the final document are defined in the \texttt{create\_master\_latex} method within the \texttt{latex\_content} string (lines 340-344).

\begin{lstlisting}[language=tex]
\pagestyle{fancy}
\fancyhf{}
% \rhead defines the right header
\rhead{\thepage{} / \pageref{LastPage}} 
% \cfoot defines the center footer
\cfoot{\includegraphics[height=0.8cm]{ideas_logo.png} ...}
\end{lstlisting}

To remove the logo from the footer, simply edit the content inside \texttt{\textbackslash cfoot\{...\}}.

\subsection{Modifying Fonts}
By default, the script relies on standard LaTeX fonts or whatever \texttt{xelatex} picks up. To enforce a specific font (e.g., Arial or Times New Roman), you need to modify the \texttt{latex\_content} in \texttt{create\_master\_latex}.

\begin{lstlisting}[language=tex]
\usepackage{fontspec}
% Add this line to set a main font (must be installed on system)
\setmainfont{Times New Roman} 
\end{lstlisting}

\section{Troubleshooting}

\begin{description}
    \item[Pandoc Error:] If you see \texttt{Pandoc died with exitcode...}, likely the input DOCX is corrupted or contains elements Pandoc cannot handle. Try saving it as a simpler DOCX.
    \item[XeLaTeX Error:] If compilation fails, check the \texttt{output/master.log} file. It usually indicates missing packages or font issues.
    \item[Empty Index:] Ensure \texttt{keywords.txt} is not empty and keywords exactly match terms in the text (case-insensitive indexing is implemented but spelling matters).
\end{description}

\end{document}
